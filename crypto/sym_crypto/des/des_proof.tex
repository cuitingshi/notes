%        File: des_proof.tex
%     Created: 二 11 29 01:00 下午 2016 C
% Last Change: 二 11 29 01:00 下午 2016 C
%
\documentclass[a4paper]{ctexart}
\usepackage{amsmath}

\title{DES 解密与加密证明}
\author{Cuiting Shi}
\date{\today{}}

\begin{document}
\maketitle{}

\section{DES 解密与加密关系}

证明DES解密只需要将原本用于加密的subkeys的次序颠倒即可

首先,对于加密,已知 $ LE_i, RE_i $, 求 $ LE_{i+1}, RE_{i+1} $, 则有
\begin{equation}
  \begin{split}
  LE_{i+1} & =  RE_i; \\
  RE_{i+1} & =  LE_i \oplus Fiestel(RE_i, subkeys_i) \\
           & =  LE_i \oplus Permutation( \\
           &   SubstituteSBOX(ExpandBlock(RE_i) \oplus Subkey_i) )
  \end{split}
  \label{equ_encrypt}
\end{equation}


则对于解密,相当于已知 $ LE_{i+1}, RE_{i+1} $, 求 $ LE_i, RE_i $, 则有
\begin{equation}
  \begin{split}
  RE_i & = LE_{i+1}; \\
  LE_i & = RE_{i+1} \oplus Fiestel(RE_i, subkeys_i) \\
       & = LE_i
  \end{split}
  \label{equ_decrypt}
\end{equation}

\end{document}


