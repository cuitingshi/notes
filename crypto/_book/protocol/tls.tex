%        File: tls.tex
%     Created: 一 11 21 03:00 下午 2016 C
% Last Change: 一 11 21 03:00 下午 2016 C
%
\documentclass[a4paper]{ctexart}

\title{\heiti{}TLS协议1.2学习札记}
\author{\kaishu Cuiting Shi}
\date{\today{}}

\bibliographystyle{plain}

\usepackage[a4paper, centering, scale=0.8]{geometry}
\usepackage[nottoc]{tocbibind}
\usepackage[]{color}
\usepackage{listings}
\definecolor{dkgreen}{rgb}{0,0.6,0}
\definecolor{mauve}{rgb}{0.58,0,0.72}
\definecolor{orange}{rgb}{0.92, 0.72, 0.15}
\lstdefinelanguage{golang} {
  classoffset=0,
  morekeywords={ type, struct, error, int, int8, int16, int32, int64, byte, string, bool, 
    uint, uint8, uint16, uint32, uint64, nil, func, var, map
  },  
  keywordstyle=\color{blue},
  classoffset=1,
  morekeywords={ fallthrough, go, chan, break,  default, range,
    switch, select, case, if, else, for, while, do, in, import, return, 
  }, 
  keywordstyle=\color{orange},
  classoffset=2,
  morekeywords={ len, copy, panic, make },
  keywordstyle=\color{red},
  classoffset=0,
  sensitive=false,
  morecomment=[l][\color{dkgreen}]{//},
  morecomment=[s][\color{dkgreen}]{/*}{*/},
  morestring=[b]``, 
}
\lstset{
  language=golang,
  morecomment=[s][\color{dkgreen}]{/*}{*/},
  stringstyle=\color{mauve},
  aboveskip=3mm,
  belowskip=3mm,
  columns=flexible,
  breaklines=true,
  basicstyle={\small \ttfamily}, 
  numbers=left,
  numberstyle={\tiny}
}

\begin{document}

\maketitle{}
\begin{abstract}
  这是一篇关于TLS协议1.2的学习笔记
\end{abstract}
\tableofcontents{}

\section{Introduction}
TLS协议是基于SSL 3.0 协议的,二者之间的差异并不是很明显.
TLS协议的主要目的是为两个通信进程提供隐私性即数据完整性. 该协议有两层: TLS Record Protocol 和 TLS Handshake Protocol.
TLS Record Protocol 是TLS的协议的最底层,但是是位于一些传输协议的上层的(比如TCP). 
TLS Record Protocol 为连接提供了安全保障,该连接具有两个主要特点:

\begin{itemize}
    \zihao{-5}\kaishu 
  \item 连接具有隐私性。其中采用了对称加密对数据进行加密(比如AES, RC4 等待). 每次连接都会生成新的对称加密的秘钥,
    该秘钥是基于TLS Handshake Protocol 生成的一个secret 生成的(当然,也可以采用除了TLS Handshake Protocol之外的其它协议进行协商获取secret). 
    TLS Record Protocol 其实也可以不适用加密, 而是直接传输明文数据.
  \item 第二个特点是连接是可靠的. 消息的传输包括使用a keyed MAC 来对消息的完整性进行校验. 其中,MAC 计算会采用类似于SHA-1 的 secure hash functions.
    当然, TLS Record Protocol 也可以不适用MAC,但是只会在如下场景中应用 -- 即另外一个协议仅仅是把TLS Record Protocol 当做协商安全参数的传输工具😂.
\end{itemize}

好吧,大家经常是使用TLS Record Protocol 来封装各种高层协议. 比如, TLS Record Protocol 可以用来封装 TLS Handshake Protocol, 从而使得在application
protocol 传输或者接收数据的首字节之前, 便可以允许server 和 client 端进行认证和协商加密算法、cryptographic keys. TLS Handshake protocol 提供的安全连接
具有下面三个基本特征: 

\begin{itemize}
    \zihao{-5}\kaishu 
  \item client 或者server 的身份可以使用公钥、crpytography (比如, RSA, DSA 等等). 该认证是可选的, 但通常至少有一方是要进行认证的.
  \item 对于共享的secret的协商过程是安全的
  \item 协商过程是可靠的, 因为只要攻击者试图修改协商的信息,通信的双方必定会觉察到的。
\end{itemize}

TLS 是独立于应用协议的,高层的协议可以透明地建立在TLS协议上。然而,由于TLS 标准并没有指定protocols 应该如何通过TLS 来增强安全性,
这使得设计者以及运行在TLS 之上的protocols 的实现必须得自己考虑如何Initiate TLS handshaking、如何解析authentication certificates.

\subsection{TLS协议1.2 VS 1.1}
相比于TLS协议1.1, TLS协议1.2 具有更大的灵活性,改进如下:
\begin{description}
    \zihao{-5}\kaishu 
  \item[-] 原先在伪随机函数PRF中所使用的 MD5/SHA-1 组合被替换成了 cipher-suite-specified PRFs
  \item[-] 数字签名元素中的MD5/SHA-1组合被替换成大衣的哈希函数, 元素现在可以增加一个field来致命哈希算法
  \item[-] 在发送完certificate\_request之后, 如果没有可用证书的话,clients必须发送空的证书列表
  \item[-] cipher suite的实现者必须得实现TLS\_RSA\_WITH\_AES\_128\_CBC\_SHA
  \item[-] cipher suites 添加了 HMAC-SHA256, 移除掉了 IDEA 和 DES
\end{description}


\section{TLS协议的目的} \label{sec:goal}
TLS协议的目的在于:
\begin{enumerate}
    \zihao{-5}\kaishu 
  \item Crytographic security: 应该使用TLS 为双方的通信建立起安全连接
  \item Interoperability: 在不知道对方的代码的情况下,不同的程序员在他们编写的应用程序中都使用了TLS, 
    那么这些应用程序是可以成功交换cryptographic parameters 的.
  \item Extensibility: TLS 的目的在于提供一种新的公钥和bulk encryption methods 可以并入的协议框架中, 
    这样可以避免实现一个全新的安全库的可能性.
  \item Relative efficiency: 由于Cryptographic operations常常非常耗费CPU, 特别是其中的public key operations, 为此,
    TLS 协议增加了session caching scheme, 从而可以减少从零开始建立连接的数量.

\end{enumerate}


\section{Cryptographic Attributes}
总共有五种cryptographic operations:
\begin{enumerate}
    \zihao{-5}\kaishu
  \item digital signing
  \item stream cipher encryption
  \item block cipher encryption
  \item authenticated encryption with additional data (AEAD) encryption
  \item public key encryption
\end{enumerate}
注意对称加密主要有两种模式, 一种是序列加密(亦称为流加密 strema cipher encryption), 另外一种是分组加密(即block cipher encryption).

%\begin{lstlisting}[frame=single, stepnumber=5, firstnumber=25, caption={ TLS package 中对于net.Conn接口的实现}]{/usr/local/Cellar/go/1.7/libexec/src/crypto/tls/conn.go}

%\end{lstlisting}

\lstinputlisting[frame=single, stepnumber=3, firstnumber=1, caption={ TLS package 中对于net.Conn接口的实现}]{/usr/local/Cellar/go/1.7/libexec/src/crypto/tls/conn.go}

\subsection{signature}
主要有两种数字签名算法 DSA: 一种RSA, 另外一种ECDSA.
对于RSA  签名, 会使用RSASSA-PKCS1-v1\_5 签名机制,其中DigestInfo必须是DER编码的.
对于DSA 签名, SHA-1 hash的20个字节是直接通过数字签名算法运行得到,会生成r 和 s 这两个大整数.

\subsection{encyption}
对于 stream cipher encryption, 明文的每一个比特位都会和由一个cryptographically secure keyed pseudorandom number generator 生成的数据(比特位数与明文的位数相同)进行异或.
换句话说, 序列密码法是一种文本加密方法,其中,在数据流的各个二进制数字中应用加密秘钥以及算法,每次处理一个比特位;

对于 block cipher encryption, 即分组密码法, 明文的每一个分组(块)都被加密成一个密文块. CBC (Cipher Block Chaining) 模式中会完成所有的block cipher encryption, 
而且 all items that are block-ciphered will be an exact multiple of the cipher block length.

对于 AEAD encryption, 明文在被加密的同时保证了明文的数据完整性. 输入可以是任意长度的, 然后 AEAD-ciphered output 通常要比输入的长度大一些,毕竟它加上了完整性校验值.

对于 public key encryption, 使用公钥算法可以对明文进行加密, 但是得到的密文只能被对应的秘钥进行解密. 
A public-key-encrypted element is encoded as an opague vector $<0..2^{16}-1>$, 其中长度由加密算法和key决定. 

而对于 RSA encryption, 这个是使用 RSAES-PKCS1-v1\_5 encryption scheme 实现的.

\section{HMAC and the Pseudorandom Function}
为了保护消息的完整性, TLS record layer 使用了 a keyed Message Authentication Code (MAC). 而该RFC中定义的 cipher suites 使用了
HMAC 来生成 MAC . HMAC 是基于哈希函数的.

另外, 为了生成 key 或者进行验证, 需要使用 construction 来将 secrets 扩展到数据块中。这个伪随机函数(PRF)的输入是 a secret, a seed, an identifying label,
输出是长度不固定的数据.

下面将基于 HMAC 来定义一个伪随机函数 PRF, 该函数还统一使用了SHA-256哈希函数. 当然,也可以通过明确指出PRF来定义新的cipher suites, 但是这得用比SHA-256安全性更高的哈希函数.

首先,定义一下P\_hash(secret, data) 函数, 该函数是用来扩张数据(其实是增加输出的MAC的位数的), 它使用一个函数函数来讲secret 和 seed 扩张为任意长度的输出:
\begin{verbatim}
P_Hash(secret, seed) = HMAC_hash(secret, A(1) + seed) + 
                       HMAC_hash(secret, A(2) + seed) +
                       HMAC_hash(secret, A(3) + seed) + ...
                       
//其中, A()的定义如下:
A(0) = seed
A(1) = HMAC_hash(secret, A(i-1))
\end{verbatim}
其中,P\_hash函数可以被调用任意次数,从而可以生成特定长度的数据. 比如,P\_Hash使用了SHA256,则如果要生成80字节的数据的话,
得迭代三次(SHA256生成的哈希是16字节,$16 + 32 + 48 = 96$), 得到96字节的数据,
然后去掉末尾的16字节, 便可以得到80字节的数据.
而TLS 中的PRF 是通过P\_Hash() 函数生成的,如下
\begin{verbatim}
//其中label是一个ASCII编码的字符串
PRF(secret, label, seed) = P_[hash](scret, label + seed)
\end{verbatim}

下面是go的tls包对于P\_hash函数的实现,注意,增加了一个新的参数hash,把其中要用到的哈希函数作为参数传进来
\begin{lstlisting}[frame=single, stepnumber=5, firstnumber=1, caption={go中对于pHash的实现}]
// pHash implements the P_hash function, as defined in RFC 4346, section 5.
func pHash(result, secret, seed []byte, hash func() hash.Hash) {
	h := hmac.New(hash, secret)
	h.Write(seed)
	a := h.Sum(nil)

	j := 0
	for j < len(result) {
		h.Reset()
		h.Write(a)
		h.Write(seed)
		b := h.Sum(nil)
		todo := len(b)
		if j+todo > len(result) {
			todo = len(result) - j
		}
		copy(result[j:j+todo], b)
		j += todo

		h.Reset()
		h.Write(a)
		a = h.Sum(nil)
	}
}
\end{lstlisting}

接下来是关于PRF算法的, PRF算法是用来从master secret来生成keys的. go的tls包对于PRF函数的实现,总共有三种 -- prf10、prf12、prf30,分别对应TLS 1.0, TLS 1.2 以及SSL/3.0协议,
下面只介绍prf12的实现:
\begin{lstlisting}[frame=single, stepnumber=5, firstnumber=1, caption={prf12的实现}]
// prf12 implements the TLS 1.2 pseudo-random function, as defined in RFC 5246, section 5.
func prf12(hashFunc func() hash.Hash) func(result, secret, label, seed []byte) {
	return func(result, secret, label, seed []byte) {
		labelAndSeed := make([]byte, len(label)+len(seed))
		copy(labelAndSeed, label)
		copy(labelAndSeed[len(label):], seed)

		pHash(result, secret, labelAndSeed, hashFunc)
	}
}
\end{lstlisting}



然后这个是TLS中对于HMAC的使用,
\begin{lstlisting}[frame=single, stepnumber=5, firstnumber=1, caption={TLS中Hmac的使用}]
func (c *Conn) encryptTicket(state *sessionState) ([]byte, error) {
	serialized := state.marshal()
	encrypted := make([]byte, ticketKeyNameLen+aes.BlockSize+len(serialized)+sha256.Size)
	keyName := encrypted[:ticketKeyNameLen]
	iv := encrypted[ticketKeyNameLen : ticketKeyNameLen+aes.BlockSize]
	macBytes := encrypted[len(encrypted)-sha256.Size:]

	if _, err := io.ReadFull(c.config.rand(), iv); err != nil {
		return nil, err
	}
	key := c.config.ticketKeys()[0]
	copy(keyName, key.keyName[:])
	block, err := aes.NewCipher(key.aesKey[:])
	if err != nil {
		return nil, errors.New("tls: failed to create cipher while encrypting ticket: " + err.Error())
	}
	cipher.NewCTR(block, iv).XORKeyStream(encrypted[ticketKeyNameLen+aes.BlockSize:], serialized)

	mac := hmac.New(sha256.New, key.hmacKey[:])
	mac.Write(encrypted[:len(encrypted)-sha256.Size])
	mac.Sum(macBytes[:0])

	return encrypted, nil
}

\end{lstlisting}

下面是crypto/hmac对于HMAC的实现源码:
\begin{lstlisting}[frame=single, stepnumber=5, firstnumber=1, caption={Go中Hmac的实现}]
type hmac struct {
	size         int
	blocksize    int
	opad, ipad   []byte
	outer, inner hash.Hash
}

func (h *hmac) Sum(in []byte) []byte {
	origLen := len(in)
	in = h.inner.Sum(in)
	h.outer.Reset()
	h.outer.Write(h.opad)
	h.outer.Write(in[origLen:])
	return h.outer.Sum(in[:origLen])
}

func (h *hmac) Write(p []byte) (n int, err error) {
	return h.inner.Write(p)
}

func (h *hmac) Size() int { return h.size }

func (h *hmac) BlockSize() int { return h.blocksize }

func (h *hmac) Reset() {
	h.inner.Reset()
	h.inner.Write(h.ipad)
}

// New returns a new HMAC hash using the given hash.Hash type and key.
func New(h func() hash.Hash, key []byte) hash.Hash {
	hm := new(hmac)
	hm.outer = h()
	hm.inner = h()
	hm.size = hm.inner.Size()
	hm.blocksize = hm.inner.BlockSize()
	hm.ipad = make([]byte, hm.blocksize)
	hm.opad = make([]byte, hm.blocksize)
	if len(key) > hm.blocksize {
		// If key is too big, hash it.
		hm.outer.Write(key)
		key = hm.outer.Sum(nil)
	}
	copy(hm.ipad, key)
	copy(hm.opad, key)
	for i := range hm.ipad {
		hm.ipad[i] ^= 0x36
	}
	for i := range hm.opad {
		hm.opad[i] ^= 0x5c
	}
	hm.inner.Write(hm.ipad)
	return hm
}
\end{lstlisting}


\section{TLS Record Protocol}
TLS Record Protocol 是一个分层协议,在每一层,消息可能包含长度、备注和内容等信息域。TLS Record Protocol 主要负责携带待传输的消息,
将数据分成分组、可能还会对其进行压缩、对数据生成MAC、加密、然后传输结果. 在接收到数据的时候, TLS Record Protocol会解密、验证MAC、解压、
重组分块、然后再发送给上一级的clients.

使用TLS Record Protocol的clients只要有四种协议: 
\begin{itemize}
    \zihao{-5}
  \item the TLS Handshake protocol
  \item the Alert Protocol
  \item the change cipher spec protocol
  \item the application data protocol
\end{itemize}

\subsection{Connection States}

TLS connection state 是TLS Record Protocol的运作环境. 它指明了压缩算法、加密算法、MAC算法. 下面是go的tls包关于connection state 的定义: 
\begin{lstlisting}[frame=single, stepnumber=5, firstnumber=1, caption={Go中connection的定义}]
 // ConnectionState records basic TLS details about the connection.
type ConnectionState struct {
	Version                     uint16     // TLS version used by the connection (e.g. VersionTLS12)
	HandshakeComplete           bool       // TLS handshake is complete
	DidResume                   bool       // connection resumes a previous TLS connection
	CipherSuite                 uint16     // cipher suite in use (TLS_RSA_WITH_RC4_128_SHA, ...)
	NegotiatedProtocol          string     // negotiated next protocol (from Config.NextProtos)
	NegotiatedProtocolIsMutual  bool       // negotiated protocol was advertised by server
	ServerName                  string     // server name requested by client, if any (server side only)
	PeerCertificates            []*x509.Certificate   // certificate chain presented by remote peer
	VerifiedChains              [][]*x509.Certificate // verified chains built from PeerCertificates
	SignedCertificateTimestamps [][]byte   // SCTs from the server, if any
	OCSPResponse                []byte     // stapled OCSP response from server, if any

	// TLSUnique contains the "tls-unique" channel binding value (see RFC
	// 5929, section 3). For resumed sessions this value will be nil
	// because resumption does not include enough context (see
	// https://secure-resumption.com/#channelbindings). This will change in
	// future versions of Go once the TLS master-secret fix has been
	// standardized and implemented.
	TLSUnique []byte
}
\end{lstlisting}

\subsection{影响connection states的参数 -- 术语}
注意,TLS record layer 会使用下面列出的安全参数生成
 client write MAC key, server write MAC key, 
 client write encryption key, server write encryption key, 
 client write IV, server write IV. 对于client write 参数, server 在接收和处理records的时候,会使用client write 这些参数,反之亦然。
当下面列的安全参数设置完毕,并且生成了相关的keys之后,则连接状态connection states 便可以通过转移到current states 被实例化. 
每个连接状态都会包含如下的条目:
\begin{itemize}
    \zihao{-5}
  \item compression state: 即压缩算法的当前状态
  \item cipher state: 加密算法的当前状态
  \item MAC key: 使用上面的那些安全参数生成的MAC算法的key
  \item sequence number: 读写状态各自维护了自己的sequence number
\end{itemize}
\emph{安全参数:}
\begin{description}
  \item[master secret]A 48-byte secret shared between the two peers in the connection
  \item[client random]A 32-byte value provided by the client
  \item[server random]A 32-byte value provided by the server
  \item[RPF algorithm]An algorithm used to generate keys from the master secret
  \item[bulk encryption algorithm]An algorithm to be used for bulk encryption. 必须指明算法的key size, 何种cipher (block, stream, or AEAD cipher), 
    cipher 所使用的block size, the length of explicit and implicit initialization vectors (or nonces)
  \item[MAC algorithm]An algorithm to be used for message authentication. 必须要指明算法所返回的MAC碼的大小
\end{description}


\subsection{Record layer}
The TLS record layer receives uninterpreted data from higher layers in non-empty blocks of arbitrary size.

\subsubsection{Fragmentation}
The record layer fragments informatin blocks into TLSPaintext records carrying data in chunks of $2^14$ bytes or less. 
Client message boundaries are not preserved in the record layer (比如, multiple client messages of the same ContentType MAY be
coalesced into a single TLSPlaintext record, or a single message MAY be fragmented across several records).也就是说,
多个相同内容类型的client messages 可能会被重组成一条记录TLSPlaintext record, 但是一条过大的client message也可能会被切分成多条记录。

此外,值得注意的是,由于历史原因,TLS1.0继承了SSL/3.0,版本号可能会写作3.1, 而TLS1.2对应3.3

\begin{lstlisting}[frame=single, stepnumber=5, firstnumber=1, caption={TLS协议中的TLS record 类型以及handshake 消息类型}]
const (
	VersionSSL30 = 0x0300
	VersionTLS10 = 0x0301
	VersionTLS11 = 0x0302
	VersionTLS12 = 0x0303
)

const (
	maxPlaintext    = 16384        // maximum plaintext payload length
	maxCiphertext   = 16384 + 2048 // maximum ciphertext payload length
	recordHeaderLen = 5            // record header length
	maxHandshake    = 65536        // maximum handshake we support (protocol max is 16 MB)

	minVersion = VersionTLS10
	maxVersion = VersionTLS12
)

// TLS record types.
type recordType uint8

const (
	recordTypeChangeCipherSpec recordType = 20
	recordTypeAlert            recordType = 21
	recordTypeHandshake        recordType = 22
	recordTypeApplicationData  recordType = 23
)

// TLS handshake message types.
const (
	typeHelloRequest       uint8 = 0
	typeClientHello        uint8 = 1
	typeServerHello        uint8 = 2
	typeNewSessionTicket   uint8 = 4
	typeCertificate        uint8 = 11
	typeServerKeyExchange  uint8 = 12
	typeCertificateRequest uint8 = 13
	typeServerHelloDone    uint8 = 14
	typeCertificateVerify  uint8 = 15
	typeClientKeyExchange  uint8 = 16
	typeFinished           uint8 = 20
	typeCertificateStatus  uint8 = 22
	typeNextProtocol       uint8 = 67 // Not IANA assigned
)

// TLS compression types.
const (
	compressionNone uint8 = 0
)
\end{lstlisting}

\subsubsection{Record Payload Protection}
record中携带的数据被压缩后,会经历 encryption 和 MAC functions 的处理. decryption functions 会反转此过程.
record中的MAC 也包含了一个序列号,从而可以检测到消息是否消失、重复或者是额外的.
下面是record写的源码实现:


\begin{lstlisting}[frame=single, stepnumber=5, firstnumber=1, caption={TLS协议中如何写record}]
// writeRecordLocked writes a TLS record with the given type and payload to the
// connection and updates the record layer state.
// c.out.Mutex <= L.
func (c *Conn) writeRecordLocked(typ recordType, data []byte) (int, error) {
	b := c.out.newBlock()
	defer c.out.freeBlock(b)

	var n int
	for len(data) > 0 {
		explicitIVLen := 0
		explicitIVIsSeq := false

		var cbc cbcMode
		if c.out.version >= VersionTLS11 {
			var ok bool
			if cbc, ok = c.out.cipher.(cbcMode); ok {
				explicitIVLen = cbc.BlockSize()
			}
		}
		if explicitIVLen == 0 {
			if _, ok := c.out.cipher.(cipher.AEAD); ok {
				explicitIVLen = 8
				// The AES-GCM construction in TLS has an
				// explicit nonce so that the nonce can be
				// random. However, the nonce is only 8 bytes
				// which is too small for a secure, random
				// nonce. Therefore we use the sequence number
				// as the nonce.
				explicitIVIsSeq = true
			}
		}
		m := len(data)
		if maxPayload := c.maxPayloadSizeForWrite(typ, explicitIVLen); m > maxPayload {
			m = maxPayload
		}
		b.resize(recordHeaderLen + explicitIVLen + m)
		b.data[0] = byte(typ)
		vers := c.vers
		if vers == 0 {
			// Some TLS servers fail if the record version is
			// greater than TLS 1.0 for the initial ClientHello.
			vers = VersionTLS10
		}
		b.data[1] = byte(vers >> 8)
		b.data[2] = byte(vers)
		b.data[3] = byte(m >> 8)
		b.data[4] = byte(m)
		if explicitIVLen > 0 {
			explicitIV := b.data[recordHeaderLen : recordHeaderLen+explicitIVLen]
			if explicitIVIsSeq {
				copy(explicitIV, c.out.seq[:])
			} else {
				if _, err := io.ReadFull(c.config.rand(), explicitIV); err != nil {
					return n, err
				}
			}
		}
		copy(b.data[recordHeaderLen+explicitIVLen:], data)
		c.out.encrypt(b, explicitIVLen)
		if _, err := c.write(b.data); err != nil {
			return n, err
		}
		n += m
		data = data[m:]
	}

	if typ == recordTypeChangeCipherSpec {
		if err := c.out.changeCipherSpec(); err != nil {
			return n, c.sendAlertLocked(err.(alert))
		}
	}

	return n, nil
}
\end{lstlisting}

\paragraph{Null or Standard Stream Cipher}

注意MAC是在加密前计算得到的,stream ciphter 加密了包含MAC的entire block.

\paragraph{CBC Block Cipher}

对于分组加密(比如3DES、AES), 

\subparagraph{IV}
The Initialization Vector (IV) SHOULD be chosen at random, and MUST be unpredictable. Note that in versions of TLS prior to 1.1,
there was no IV field, and the last ciphertext block of the previous reord (the ``CBC residue'') was used a the IV. This was 
changed to prevent the attacts described in CBCATT. For block ciphers, the IV length is of length 等于安全参数中的 \begin{verbatim} record_iv_lenghth \end{verbatim},
亦等于\begin{verbatim} block_size \end{verbatim}

\end{document}


\end{document}
